\documentclass[DM,lsstdraft,authoryear,toc]{lsstdoc}
% lsstdoc documentation: https://lsst-texmf.lsst.io/lsstdoc.html
\input{meta}

% Package imports go here.

% Local commands go here.

%If you want glossaries
% \input{aglossary.tex}
% \makeglossaries

\title{Guidelines for Community Participation in Data Preview 0}

% Optional subtitle
% \setDocSubtitle{A subtitle}

\author{The Community Engagement Team and the Operations Executive Team}

\setDocRef{RTN-004}
\setDocUpstreamLocation{\url{https://github.com/rubin-observatory/rtn-004}}

\date{\vcsDate}

% Optional: name of the document's curator
% \setDocCurator{The Curator of this Document}

\setDocAbstract{%
This document provides an overview of the goals and policies for community participation in the Rubin Observatory pre-operations Data Preview 0 (DP0).
}

% Change history defined here.
% Order: oldest first.
% Fields: VERSION, DATE, DESCRIPTION, OWNER NAME.
% See LPM-51 for version number policy.
\setDocChangeRecord{%
  \addtohist{1}{YYYY-MM-DD}{Unreleased.}{Melissa Graham}
}


\begin{document}

% Create the title page.
\maketitle
% Frequently for a technote we do not want a title page  uncomment this to remove the title page and changelog.
% use \mkshorttitle to remove the extra pages

\renewcommand{\thepage}{\arabic{page}}% Arabic numerals for page counter

\setcounter{page}{1}% Start page number

% ADD CONTENT HERE
% You can also use the \input command to include several content files.

\section{Introduction to Data Preview 0}\label{sec:intro}

There are three planned Data Previews (DPs): DP0 based on simulated data (2021), DP1 based on Commissioning Camera data, and DP2 based on Science Camera data.
The Data Previews enable the community to access LSST-like data on a shared-risk basis and to participate in the science validation process. 

\subsection{The Goals of DP0}\label{ssec:intro_goals}

As described in \citeds{RTN-001}, the two goals of DP0 are:
\begin{enumerate}
\item To serve as an early integration test of the Data Management System (DMS) and inform further development of the science pipelines and the Rubin Science Platform (RSP).
\item To prepare the community to use the LSST data products and services, and especially to enable early science with commissioning and year 1 data.
\end{enumerate}

In order to achieve these goals, community participation in DP0 has two main components:
(1) performing scientific exercises with the data products and services in the RSP, and
(2) taking action to inform or enhance the performance of the DMS, the RSP, or the science community.
Participants from the science community, "DP0 delegates", will be data rights holders\footnote{Data rights holders are US or Chilean scientists, or individuals associated with an International Contribution (in-kind) proposal in progress, as defined in \citeds{RDO-013}.} who represent a diversity of backgrounds.
The term "delegates" reflects the fact that these individuals will have the important role of representing the science community as learners and as contributors to the overall success of DP0.
The nature of these two goal-driven components of DP0 participation will vary according to individual delegates' interests and experience.

\subsection{Two Stages: DP0.1 and DP0.2}\label{ssec:intro_stages}

The simulated data set for DP0 was generated by the Dark Energy Science Collaboration (DESC) and is being used under an agreement between DESC and Rubin.
This data set is the second in a series of data challenges, and is thus named DC2.
More information about DC2 can be found in the DESC DC2 paper \citep{2020arXiv201005926L} and on the DESC website\footnote{\url{https://lsstdesc.org/DC2-production/Documents/}}.

The DP0.1 data set will contain the DC2 images and catalogs processed by DESC.
The DP0.1 data products will be available in the RSP in their original format (which is similar to planned LSST data products).
The DP0.1 data set will begin to be available in the RSP starting June 30 2021. This will at first include a subset of the data products (processed visit images and visit-level catalogs without full-scale validation and quality analysis) and subsequently include science-ready data products.

The DP0.2 data set will contain the DC2 images and catalogs reprocessed by the Rubin science pipelines.
The DP0.2 data products will be available in the RSP in a format that is consistent with the planned LSST data products.
The DP0.2 data set is scheduled to become available in the RSP starting March 31 2022.

The timeline for identifying and on-boarding DP0 delegates from the science community Section \ref{ssec:sel_time} is driven by this timeline for DP0 data availability in the RSP.
Further details about the DP0.1 and DP0.2 data sets will be presented elsewhere\footnote{Any references in other documents to "early access" data sets for DP0 refer to images and visit-level catalogs without full-scale validation and quality analysis. This is an internal term to guide Rubin Observatory's DP0-related development, and does not relate to "early access" for community participants.}.

\section{Delegate Benefits and Responsibilities}\label{sec:del}

As DP0 access will be a limited resource for the science community, it comes with both benefits and responsibilities.

\subsection{Benefits}\label{ssec:del_bene}

Anticipated benefits for DP0 delegates include:
\begin{itemize}
\item an accelerated learning experience in the Rubin Science Platform
\item becoming more competitive for grants and early science
\item opportunities to take leadership role in their field and/or Science Collaboration
\item having options to publish or publicize the results of their DP0 work (software, analysis)
\item the ability to advocate for RSP developments beneficial to their scientific field
\end{itemize}

On-boarding resources and organizational infrastructure provided by Rubin Observatory will be necessary for DP0 delegates to obtain the benefits. These are described in Section \ref{ssec:res_sup}.

These potential benefits of DP0 participation motivate the focus on diversity and equity in the DP0 delegate selection process in Section \ref{ssec:sel_equity}.

\subsection{Responsibilities}\label{ssec:del_resp}

The most important responsibility is that DP0 delegates learn how to use the RSP via the on-boarding process well enough to perform some basic scientific exercises, and then take action that informs or enhances the DMS, the RSP, or a Science Collaboration. 

Such actions are envisioned to be simple and optional (they will not be tracked), and it will be left to the discretion of individual delegates to decide what kind of actions best match their interests and expertise.
Examples of delegate actions are provided in Section \ref{ssec:res_act}.

The estimated time commitment for DP0 delegates to both fulfill their responsibilities and gain the benefits of participation is approximately 1-3 hours per week.
This estimate is an average over an individual's DP0 participation period, which should span three months at minimum, but may last from their on-boarding to beyond the DP0.2 timeframe. 
The time of individual delegates' will not be tracked, these are just suggestions of the minimum amount of time that is likely to be needed for a delegate to reap the benefits.

In addition to the above, all science community DP0 delegates are expected to:
\begin{itemize}
\item abide by the data rights policy in \citeds{RDO-013}
\item adhere to security policies regarding their IDF accounts
\item abide by usage policies of the RSP and the IDF
\item use the on-boarding resources to learn about the RSP
\item follow the off-boarding steps if/when appropriate
\item engage in DP0-related discussions via the Community Forum when possible
\item help other DP0 delegates resolve issues when possible
\end{itemize}

Policies regarding account usage, security, and off-boarding are described in Section \ref{ssec:res_pol}, and on-boarding resources for DP0 delegates in Section \ref{ssec:res_sup}. 


\section{Delegate Selection Process}\label{sec:sel}

There will be 300 IDF user accounts available for science community DP0 delegates, in addition to the ~150 accounts for Rubin staff members.
This marks an almost order of magnitude increase in the number of community accounts in the current NCSA implementation of the RSP (via the Stack Club).

The number of IDF accounts remains constrained during the Rubin Observatory construction and pre-operations phase -- and especially for DP0 -- due to the Rubin pre-operations team's limited scope in providing support for data software and services that are still in development.
Time is needed to scale up in a sustainable manner.
At least by the start of Rubin Observatory Operations, all data rights holders will be able to have RSP accounts.
It is a primary goal of the Rubin pre-operations team to provide RSP access as soon as possible to all data rights holders. 

All DP0 delegates must be data rights holders, e.g., US or Chilean scientists, or individuals associated with an International Contribution (in-kind) proposal in progress \citeds{RDO-013}.

\subsection{Delegate Groups}\label{ssec:sel_grps}

Achieving the goals of DP0 requires a broad diversity in the DP0 delegates, each with their own experience, expertise, or novice perspective, and each representing and reporting back to different parts of the broader community.

The primary goal of this application process is to ensure that the final set of DP0 delegates includes the full range of diversity in the science community, provides the beneficial opportunities of DP0 participation equitably, and meets the Rubin pre-operations team's needs.

To guide the selection process, the groups of DP0 delegates are defined in Table \ref{tab:delegate_groups}, and the \emph{approximate} number of delegates in each group estimated.

\begin{table}[!h]
\centering
\caption{Approximate targeted distribution for diversity-based DP0 delegate groups.}\label{tab:delegate_groups}
\begin{tabular}{lll}
\hline
Group & $\sim$\% & Description \\
\hline \hline
A & 10 & experienced users/builders of the DESC DC2 simulated data set \\
B & 10 & experienced users of science platforms (e.g., RSP, SciServer, DataLab) \\
C & 10 & representatives of teams building iDACs (independent Data Access Centers) \\
D & 10 & scientific expertise in Rubin science pillars, deep imaging, big data, etc. \\
E & 10 & representatives of small and/or underserved US institutions and colleges \\
F & 10 & self-identified novice-level learners and students (and students' advisors) \\
G & 10 & representatives of the Chilean astronomy community \\
H & 30 & representatives of the Science Collaborations \\
\hline
\end{tabular}
\end{table}

Representatives in all groups have the same expectations and responsibilities, as in Section \ref{ssec:del_resp}.
The selection committee described in Section \ref{ssec:sel_comm} will consider all applicants equally for all groups with which applicants self-identify, using the diversity-based selection criteria in Section \ref{ssec:sel_crit}.
Applicants will be able self-identify with multiple groups on the application form (Section \ref{ssec:sel_form}).
Specifically, please note that Chilean astronomers and Science Collaboration members are {\it not} restricted to gaining a delegate position via Groups G and H. 

\subsection{Selection Committee}\label{ssec:sel_comm}

The delegate selection committee will be facilitated by the Rubin Observatory Community Engagement Team (CET), and will contain representatives from (or identified by) the following sources.
Individual committee members will be identified by the CET and the Rubin directorate.

\begin{table}[!h]
\centering
\caption{Approximate targeted distribution of selection committee representation.}\label{tab:selection_committee}
\begin{tabular}{ll}
\hline
Number & Source \\
\hline \hline
4 & Science Advisory Council (SAC) and the LSST Science Collaborations \\
3 & Community Engagement Team and Rubin Observatory Director's Office \\
3 & Rubin Operations (e.g., Data Production, SPaRE, RSP, In-Kind Program Coordinators) \\
2 & NOIR Lab (e.g., Community Science Data Center), DOE, or other institutions \\
2 & Chilean astronomical community \\
\hline
\end{tabular}
\end{table}

To keep committee participation manageable, committee members will be primarily tasked with providing shortlists of applicants for one or more groups.
For example, committee members from the SAC and Science Collaborations would be primarily tasked with generating a shortlist for Group H, but would also provide input on the other groups.
The CET and the Rubin directorate will merge these shortlists into a final list of DP0 delegates.

All selection committee members will use the diversity-based selection criteria in Section \ref{ssec:sel_crit} when forming shortlists for the various groups.
Prior to meeting, the CET will facilitate a session on unconscious bias for the selection committee members.
The capacity to view form results without names will be provided to the committee.

\subsection{Selection Criteria}\label{ssec:sel_crit}

Diversity in representation is the top priority.
The groups described in Section \ref{ssec:sel_grps} are designed to achieve a diverse population of DP0 delegates in several facets: experience, expertise, career level, institution type, geographical location, and scientific field.
The selection committee will also consider these aspects of diversity within the groups.

Short-form write-in responses will be solicited from applicants so that they may explain their relationship to the groups with which they self-identify.
However, applicants will not be asked to, e.g., provide a robust scientific justification for their participation in DP0, or a detailed plan for their DP0 delegate action (Section \ref{ssec:res_act}).
While it might at first seem reasonable to consider aspects such as "science impact" or "contributed value" in a selection process like this, doing so could run counter to diversity initiatives (i.e., not all applicants will be equally able to express such aspects in language familiar to the committee members).
In a supportive environment with adequate resources (Section \ref{ssec:res_sup}), all DP0 delegates should have an equivalent potential to succeed in their endeavors.

\subsection{Equitable Access}\label{ssec:sel_equity}

There are 3 main ways in which equitable access to DP0 is being supported: broad advertising, a staged application process, and on-boarding resources.

Broad advertising is necessary for delegates to self-identify as potential participants, and for reaching potential delegates who are not yet actively participating in Rubin science.
The CET will work together with the Rubin Observatory communications team to ensure the call for applications is broadly advertised. 
The CET will also work with the Rubin directorate to identify individuals in groups A through E, and reach out directly to encourage applications. 

A staged application process is implemented to maximize the inclusion of delegates who would not be able or willing to maintain the DP0 time commitment over both DP0.1 and DP0.2, and because it is recognized that it might take time to achieve a diverse set of DP0 delegates and to provide equitable access to DP0 (Section \ref{ssec:sel_time}).

On-boarding resources and a supportive community will be generated and maintained by the CET (Section \ref{ssec:res_sup}).
Such resources are recognized as necessary for delegates to acquire the benefits of DP0 participation, regardless of their past experience or expertise. 

\subsection{Application Form Contents}\label{ssec:sel_form}

A diversity-based selection process requires that adequate appropriate information be collected from applicants.
The information on the application form will also be used by the CET to help design the delegate resources described in Sections 4.2 and 4.3. 

The application form should be self-explanatory -- applicants should not \emph{have to} refer to this document (or any other) in order to complete the form -- but also link to additional information about DP0.
The application form should include a preamble with the deadline for consideration, and information for a point-of-contact in case of questions.

Below is a proposed list of the data that should be collected.
All of the drop-down menus should have multiple selections enabled.

Personal Information
\begin{itemize}
\item first name, surname (2 write-in fields)
\item pronouns (drop-down menu with write-in option)
\item affiliation(s) (drop-down menu with write-in option)
\item city, country (2 write-in fields)
\item email address (1 write-in field)
\item first and second languages, preferred correspondence language (3 drop-down menus)
\item career stage (drop-down menu with write-in option)
\end{itemize}

DP0-Related Information
\begin{itemize}
\item science keywords (1 short-form write-in field)
\item Science Collaboration (drop-down menu that includes 'none')
\item delegate group identification (drop-down menu that includes 'none')
\item delegate group relation (one short form write-in field per group, plus one for 'none')
\item group F plus-one request (a yes/no option for applicants to indicate if they are a student/advisor in need of an additional RSP account for their advisor/student).
\end{itemize}

Applicant Confirmations
\begin{itemize}
\item time commitment (yes/no to confirm recognition of minimum expected time commitment)
\item delegate responsibilities (yes/no to confirm recognition of expectations and responsibilities)
\item data rights (yes/no/idk to confirm data rights)
\end{itemize}


\subsection{Selection Timeline}\label{ssec:sel_time}

The application form will remain open, accumulating submissions, for the duration of DP0.
Applications will be reviewed by the selection committee twice: once in advance of DP0.1 and then again in advance of DP0.2 (as in Table \ref{tab:selection_timeline}).

To ensure that at least some accounts are available for new people to join for DP0.2, ~25\% of the 300 total DP0 accounts will not be allocated prior to DP0.2.
The following paragraph provides additional motivation for this multi-stage approach to accounts allocation.

The entire duration of DP0 is long, at least 1 year and potentially longer.
It is anticipated that many delegates would not be able or willing to maintain the DP0 time commitment (Section \ref{ssec:del_resp}) over that long of a time frame, and that some scientists might be more interested in the DP0.2 data set than DP0.1.
Additionally, achieving diverse and equitable access to DP0 might also take some time: the advertisements must percolate, people must free up time in their schedules or obtain grants to fund their work, and the DP0-related resources and support network will become enriched over time (and thus more attractive to a wider variety of participants).
To avoid the perception that access to DP0 is restricted to individuals who can commit, up-front, to regular participation for more than one year, 25\% of the 300 DP0 accounts will be reserved for DP0.2.
Any accounts relinquished by DP0.1 participants will be added to this reservation and reallocated for DP0.2.

Applications might also be reviewed on intermediate timescales in order to more fully include members of the groups identified in Table \ref{tab:delegate_groups} and achieve the diversity-related goals of DP0 participation.
On such intermediate timescales the applications might be reviewed only by the Rubin Operations executive team and/or the Community Engagement Team instead of reconvening a full selection committee. 

\begin{table}[!h]
\centering
\caption{Projected DP0 delegate application process timeline.}\label{tab:selection_timeline}
\begin{tabular}{lllll}
\hline
Stage & \% (of 300) & Applications Due & Delegates Notified & Data Available \\
\hline \hline
DP0.1 & 75\%    &  Apr 30 2021   &  May 31 2021  &  Jun 30 2021 \\
DP0.2 & 100\%  &  Jan 31 2022   &  Feb 28 2022  &  Mar 31 2022 \\
\hline
\end{tabular}
\end{table}


\section{Resources for DP0 Delegates}\label{sec:res}

\subsection{Policies for IDF Accounts}\label{ssec:res_pol}

For the duration of DP0, the RSP will be installed in the Interim Data Facility (IDF) deployed on the Google Cloud Platform.
More information about the IDF and its installation will be available elsewhere.
Access to the RSP and the DP0 data products and services are limited to data rights holders, and all DP0 delegates must abide by the data rights policies in \citeds{RDO-013}.

All DP0 delegates will be asked to abide by specific policies related to account authentication, security, use, and deactivation.
The full details for these policies will be provided to delegates prior to account creation. 

Delegates may not share accounts or passwords, but will be free (and encouraged!) to provide tours of the RSP or live demonstrations of their work to individuals who are not DP0 delegates.
Account usage policies might include, e.g., best practices for starting and stopping servers, understanding that computational processes which overburden the system might be curtailed, abiding by storage capacity quotas, and following on- and off-boarding steps.

As the total number of accounts will be limited to 300, delegates who don't appear to be using their accounts might be periodically contacted and asked to consider voluntarily deactivating to make way for others.
In the case of a high oversubscription for DP0 participation, account holders who do not respond to such inquiries might have their access suspended to make way for others.
Policies related to account suspension will be clarified for DP0 delegates during on-boarding.
Note that this is a temporary measure for the Data Previews, and not how accounts will work during Operations.

In particular, delegates who are given accounts at the start of DP0.1 but who do not have the work-hours to start using them for a few months (due to prior commitments, teaching loads, etc.) will be accommodated. 
Active, responsive delegates who are given accounts for DP0.1 may keep them, and will not need to reapply for DP0.2.

\subsection{Support for Delegates}\label{ssec:res_sup}

It is recognized that on-boarding resources and organizational infrastructure provided by Rubin Observatory will be necessary to enable equitable access for all delegates including novice science platform users (Section \ref{ssec:sel_equity}), and for DP0 delegates to obtain the benefits (Section \ref{ssec:del_bene}), fulfill the responsibilities (Section \ref{ssec:del_resp}) -- in particular, delegate actions (Section \ref{ssec:res_act}). 

The full set of resources will be presented elsewhere, but would include, e.g., virtual live sessions and recordings of RSP demonstrations, user guides and documentation, and jupyter notebook tutorials.
The organizational infrastructure will include online open forums for question-and-answer, issue resolution, and general discussion (e.g., Community.lsst.org), as well as regularly scheduled working groups and "office hours" run by Rubin staff.
Opportunities will be provided for DP0 delegates to communicate with the CET about needed resources and support (e.g., periodic surveys, drop-in discussion sessions).

DP0 delegates are responsible for engaging with the resources in order to learn about the RSP and the DP0 data set (Section \ref{ssec:del_resp}).
Creating or enhancing DP0 resources and support infrastructure -- e.g., writing tutorial notebooks, leading a collaborative working group, helping other delegates resolve issues -- are all potential delegate actions (Section \ref{ssec:res_act}).

\subsection{Example Delegate Actions}\label{ssec:res_act}

As described in Section \ref{ssec:del_resp}, the most important responsibility of DP0 delegates is to learn how to do scientific analyses in the RSP, and then take action that informs or enhances the DMS, the RSP, or a Science Collaboration. 

The objective is for these delegate actions to extend the benefits of DP0 beyond the limited number of DP0 participants that Rubin Observatory is able to support, and to maximize the impact of DP0 with respect to the goals of testing and informing development of the RSP and preparing the scientific community for early science (Section \ref{ssec:intro_goals}). 
These actions are voluntary and will not be tracked.
Group work is encouraged.
It is left to individual delegates to decide what kind of actions best match their interests and expertise. 

Examples of potential delegate actions include but are not limited to:
\begin{itemize}
\item participate in feedback surveys managed by Rubin Observatory
\item provide a public Jupyter notebook demo of some science analysis
\item small written reports on activities (e.g., in Community Forum)
\item short presentation on activities during topical DP0-related virtual workshops
\item video demonstrations of "how-to" with the RSP
\item work through an "RSP Test Checklist" provided by Rubin Observatory
\item leading working groups composed of other DP0 delegates
\item helping to resolve reported issues or support other delegate's work
\end{itemize}


\appendix
% Include all the relevant bib files.
% https://lsst-texmf.lsst.io/lsstdoc.html#bibliographies
\section{References} \label{sec:bib}
\renewcommand{\refname}{} % Suppress default Bibliography section
\bibliography{local,lsst,lsst-dm,refs_ads,refs,books}

% Make sure lsst-texmf/bin/generateAcronyms.py is in your path
\section{Acronyms}
\input{acronyms.tex}
% If you want glossary uncomment below -- comment out the two lines above
%\label{sec:acronyms}
%\printglossaries


% \input{checklist}
\end{document}
