\documentclass[DM,authoryear,toc]{lsstdoc}
% lsstdoc documentation: https://lsst-texmf.lsst.io/lsstdoc.html
\input{meta}

% Package imports go here.

% Local commands go here.

%If you want glossaries
% \input{aglossary.tex}
% \makeglossaries

\title{Guidelines for Community Participation in Data Preview 0}

% Optional subtitle
% \setDocSubtitle{A subtitle}

\author{Melissa Graham, the Community Engagement Team, and the Operations Executive Team}


\setDocRef{RTN-004}
\setDocUpstreamLocation{\url{https://github.com/rubin-observatory/rtn-004}}

\date{\vcsDate}

% Optional: name of the document's curator
% \setDocCurator{The Curator of this Document}

\setDocAbstract{%
This document provides an overview of the goals and policies for community participation in the Rubin Observatory pre-operations Data Preview 0 (DP0).
}

% Change history defined here.
% Order: oldest first.
% Fields: VERSION, DATE, DESCRIPTION, OWNER NAME.
% See LPM-51 for version number policy.
\setDocChangeRecord{%
  \addtohist{1}{YYYY-MM-DD}{Unreleased.}{Melissa Graham}
}


\begin{document}

% Create the title page.
\maketitle
% Frequently for a technote we do not want a title page  uncomment this to remove the title page and changelog.
% use \mkshorttitle to remove the extra pages

\renewcommand{\thepage}{\arabic{page}}% Arabic numerals for page counter

\setcounter{page}{1}% Start page number

% ADD CONTENT HERE
% You can also use the \input command to include several content files.

 

\appendix
% Include all the relevant bib files.
% https://lsst-texmf.lsst.io/lsstdoc.html#bibliographies
\section{References} \label{sec:bib}
\renewcommand{\refname}{} % Suppress default Bibliography section
\bibliography{local,lsst,lsst-dm,refs_ads,refs,books}

% Make sure lsst-texmf/bin/generateAcronyms.py is in your path
\section{Acronyms}
\input{acronyms.tex}
% If you want glossary uncomment below -- comment out the two lines above
%\label{sec:acronyms}
%\printglossaries


% \input{checklist}
\end{document}
