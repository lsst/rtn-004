\documentclass[DM,lsstdraft,authoryear,toc]{lsstdoc}
% lsstdoc documentation: https://lsst-texmf.lsst.io/lsstdoc.html
\input{meta}

% Package imports go here.

% Local commands go here.

%If you want glossaries
% \input{aglossary.tex}
% \makeglossaries

\title{Guidelines for Community Participation in Data Preview 0}

% Optional subtitle
% \setDocSubtitle{A subtitle}

\author{The Community Engagement Team and the Operations Executive Team}

\setDocCurator{Melissa Graham}
\setDocRef{RTN-004}
\setDocUpstreamLocation{\url{https://github.com/rubin-observatory/rtn-004}}

\date{\vcsDate}

% Optional: name of the document's curator
% \setDocCurator{The Curator of this Document}

\setDocAbstract{%

In 2021 the Rubin Observatory began Data Preview 0 (DP0), the first of three data previews during the period leading up to the start of Rubin Operations.
DP0 makes simulated LSST-like data products, which are similar to the future LSST data releases, available in the Rubin Science Platform (RSP).
The goals of DP0 are to serve as an early integration test of the LSST science pipelines and the RSP and to enable the community to prepare to do science with the unprecedented LSST data set. 
To meet these goals, the Rubin Observatory will support 600 RSP accounts for "DP0 delegates": individuals who will represent the science community as early users of the RSP.
DP0 delegates benefit from an accelerated learning experience with the RSP's query, visualization, and analysis tools; are free to pursue their own science with the DP0 data set in the RSP; and have opportunities to provide feedback about their experience to Rubin Observatory.
As the number of RSP accounts for community access to DP0 is limited, a set guidelines for how Rubin Observatory should select the participants is required. 
This document describes the diversity-driven application and selection process for DP0 delegates, and briefly describes some of the resources, support, and activities that will be available to them.
}

% Change history defined here.
% Order: oldest first.
% Fields: VERSION, DATE, DESCRIPTION, OWNER NAME.
% See LPM-51 for version number policy.
\setDocChangeRecord{%
  \addtohist{1}{2020-12-18}{Released for circulation and feedback.}{Melissa Graham}
  \addtohist{2}{2021-01-27}{Released.}{Melissa Graham}
  \addtohist{3}{2021-12-xx}{Updated prior to DP0.2.}{Melissa Graham}
}


\begin{document}

% Create the title page.
\maketitle
% Frequently for a technote we do not want a title page  uncomment this to remove the title page and changelog.
% use \mkshorttitle to remove the extra pages

\renewcommand{\thepage}{\arabic{page}}% Arabic numerals for page counter

\setcounter{page}{1}% Start page number

% ADD CONTENT HERE
% You can also use the \input command to include several content files.

\section{Introduction to Data Preview 0}\label{sec:intro}

There are three planned Data Previews (DPs) during the period leading up to the start of Rubin Operations, termed pre-operations.
DP0 is based on simulated data.
DP1 and DP2 will be based on data taken during Rubin Observatory commissioning; DP1 based on data taken with the Rubin Observatory Commissioning Camera (ComCam), and DP2 based on data taken with the LSST Camera (LSSTCam) -- the science instrument for Rubin Observatory.
DP0 started on June 30 2021 and will continue through 2022, with DP1 and DP2 currently foreseen to begin in late 2022 and 2023 respectively.
The Data Previews will enable the community to access and analyze LSST-like data in the Rubin Science Platform (RSP) on a shared-risk basis. 

\subsection{The Goals of DP0}\label{ssec:intro_goals}

As described in \citeds{RTN-001}, the two goals of DP0 are:
\begin{enumerate}
\item To serve as an early integration test of the Data Management System (DMS) and inform further development of Rubin's LSST Science Pipelines (hereafter LSST Science Pipelines) and the Rubin Science Platform (RSP).
\item To enable the community to learn to use the LSST data products and services, to prepare for early science with commissioning and year 1 data, and to provide feedback to Rubin Observatory. 
\end{enumerate}

In order to achieve these goals, community participation in DP0 has two main components:
(1) performing scientific exercises with the data products and services in the RSP, and
(2) taking action to inform or enhance the performance of the DMS, the RSP, or the science community.

Participants from the science community, "DP0 delegates", are data rights holders\footnote{Data rights holders are US or Chilean scientists, or individuals associated with an International Contribution (in-kind) proposal in progress, as defined in \citeds{RDO-013}.} who represent a diversity of backgrounds.
The term "delegates" reflects the fact that these individuals are representing the science community as learners and as contributors to the achievement of the DP0 goals, as an essential step along the path towards the success of the LSST. 

The nature of the two goal-driven components of DP0 participation listed above will vary according to individual delegates' interests and experience.

\subsection{The Total Number of DP0 Participants}

There will be 600 RSP user accounts at the IDF available for science community DP0 delegates.
In addition, there will be $\sim$150 accounts to cover all Rubin Observatory staff members who need one (e.g., Data Management, Verification and Validation, Commissioning, and Community Engagement team members).
For reference, 600 RSP accounts for community participants marks an order of magnitude increase in the number of community accounts in the current NCSA instance of the RSP (via the Stack Club).

Of these 600 accounts, 300 were allocated in 2021 in advance of the release of DP0.1 on June 30 2021 (Section \ref{sec:sel}).
Another 300 accounts will be allocated in 2022 in advance of the release of DP0.2 on June 30 2022 (Section \ref{sec:sel2}).

The number of RSP accounts remains constrained during the Rubin Observatory construction and pre-operations phase -- and especially for DP0 -- due to the Rubin pre-operations team's limited scope in providing support for data products, software, and services that are still in development.
An order of magnitude increase represents a safe and sustainable scaling up of the Data Management System, which is also still in development, at the Interim Data Facility.
Time is needed to scale up in a sustainable manner.

It is a primary goal of the Rubin pre-operations team to provide RSP access as soon as possible to all data rights holders over the course of the data previews. 
At least by the start of Rubin Observatory Operations, all data rights holders will be able to have RSP accounts.

All DP0 delegates must be data rights holders, e.g., US or Chilean scientists, or individuals associated with an International Contribution (in-kind) proposal in progress \citeds{RDO-013}.


\subsection{The DP0 Simulated Data Set}\label{ssec:intro_dc2}

The simulated data set adopted for DP0 was generated by the Dark Energy Science Collaboration (DESC), and is being used under an agreement between the DESC and Rubin.
This data set is the second in a series of data challenges, and is thus named DC2.

As explained in the DESC's DC2 paper \citep{2020arXiv201005926L}, the core science goals that DC2 was developed to address are all extragalactic, such as weak lensing correlations and Type Ia supernova cosmology.
For these reasons, the simulation is five years of observations of 300 square degrees of extragalactic regions in six filters with a baseline survey strategy, plus one square degree of a deep drilling field.
Realistic observing conditions and a baseline survey strategy were applied (for readers familiar with the term, the operations simulation ``OpSim minion\_1016" was used).

The types of extragalactic objects simulated for DC2 include clusters, galaxies, Type Ia supernovae, and active galactic nuclei.
The types of galactic objects include stars, about 10\% of which have periodic or non-periodic variability.
DC2 does not include solar system objects, or other types of extragalactic transients such as tidal disruption events or core collapse supernovae.
A full description of the DESC's DC2 simulation can be found in \citep{2020arXiv201005926L}, and in particular the synthetic galaxy catalog is described in \citet{2019ApJS..245...26K}.

As described in Section \ref{ssec:intro_stages} below, the DP0 data will be released in two stages: DP0.1 and DP0.2.
For DP0.1, the image processing outputs that will be available will simulate the LSST data release (DR) data products for the wide-fast-deep (WFD) main survey at five years and are referred to as DR6-WFD (for LSST, DR1 is 0.5 years of data, DR2 is the first year, etc.).
DR6-WFD includes the raw images, processed visit images (PVIs, where a visit is a single 30 second observation), source catalogs for the PVIs, coadded images for each band, and multi-band object catalogs for the coadds.
It is important to note that the processing does not include difference image analysis, and that the catalog photometry is all from the direct images.
It is also important to note that the source catalogs for the PVIs are unassociated and forced photometry has not been run on the visit images, which means that there are no time-domain object catalogs or light curves.

For DP0.2 the same data products as described above will be made available, but in this case processed by the Rubin operations team using an up-to-date version of the LSST Science Pipelines provided by the Rubin Observatory construction team.
This might include additional data products that are produced by the pipelines as of Sep 2021.
For example, difference image analysis (DIA) and DIA photometry are planned as additional DC2 data products for DP0.2.

\subsection{Two Stages: DP0.1 and DP0.2}\label{ssec:intro_stages}

The DP0.1 data set contains the DC2 images and catalogs as processed by the DESC using the LSST Science Pipelines (v19.0.0).
The DP0.1 data products are available in the RSP in their original format as processed by the DESC, which is similar to that of planned LSST data products.
Because the DP0.1 data set release date was June 30 2021, the selection process for the first stage of delegates (Section \ref{ssec:sel_time}) was designed to complete by May 31 2021.

The DP0.2 data set will contain the DC2 images and catalogs reprocessed by Rubin Observatory using a more recent version of the LSST Science Pipelines than DESC used to process the DP0.1 products.
The DP0.2 data products will be available in the RSP in a format that is consistent with the planned LSST data products.
The projected date by which the DP0.2 data products will be made available to DP0 delegates in the RSP is June 30 2022. 
Therefore, in Section \ref{ssec:sel_time} the second stage of DP0 delegates are scheduled to have their accounts activated by May 31 2022.

Further details about the DP0.1 and DP0.2 data sets, such as their exact table schema and formats, will be presented as part of the resources and materials made available to DP0 delegates (Section \ref{ssec:res_sup}).

There will also be a few capability differences between the two phases, which might affect some delegates' planned activities.
For example, in DP0.1, the notebook aspect of the RSP offers image access via the Butler\footnote{The Butler is a middleware component of the DMS for persisting and retrieving image datasets.}, whereas in DP0.2 image access will be available via Virtual Observatory (VO) services.
In DP0.1, the portal aspect of the RSP is only be available for catalogs, whereas in DP0.2 it will also be available for images.
Further details about capability differences can be found in \citeds{RTN-001}, and will also be clarified in the resources and materials for DP0 delegates (Section \ref{ssec:res_sup}).


\subsection{Delegate Benefits and Responsibilities}\label{ssec:intro_del}

As DP0 access will be a limited resource for the science community, it comes with both benefits and responsibilities.

\subsubsection{Benefits}\label{sssec:intro_del_bene}

Anticipated benefits for DP0 delegates include, for example:
\begin{itemize}
\item an accelerated learning experience in the Rubin Science Platform
\item expertise that could be useful for grant applications or early science publications
\item the potential to develop, e.g., analysis software, visualization tools, for LSST science 
\item having options to publish or publicize the results of their DP0 work (software, analysis)
\item the ability to advocate for RSP developments beneficial to their scientific field
\end{itemize}

On-boarding resources and organizational infrastructure provided by Rubin Observatory will be necessary for DP0 delegates to obtain the benefits.
These are described in Section \ref{ssec:res_sup}.

These potential benefits of DP0 participation motivate the focus on diversity and equity in the DP0 delegate selection process in Section \ref{ssec:sel_equity}.

\subsubsection{Responsibilities}\label{sssec:intro_del_resp}

The most important responsibility is that DP0 delegates learn how to use the RSP, installed in the Interim Data Facility (IDF), via the on-boarding process well enough to perform some basic scientific exercises, and then take action that informs or enhances the DMS, the RSP, or a Science Collaboration. 

Such actions are envisioned to be simple and optional (they will not be tracked), and it will be left to the discretion of individual delegates to decide what kind of actions best match their interests and expertise.
Examples of delegate actions are provided in Section \ref{ssec:res_act}.

The time spent by DP0 delegates will not be tracked and there is no minimum (or maximum), but the estimated time commitment for DP0 delegates to gain at least some of the benefits of participation is approximately 12-36 hours (e.g., 1-3 hours per week over three months).
This time can be distributed in any manner that is convenient to the delegate, and allocated accounts do not expire\footnote{In the case of oversubscription for RSP accounts, delegates might be asked to relinquish their accounts if they are not using them, in order to make way for others (Section \ref{ssec:res_pol}).}.

In addition to the above, all DP0 delegates will be expected to:
\begin{itemize}
\item abide by the Rubin Observatory data policy \citedsp{RDO-013}
\item adhere to security policies regarding their RSP accounts at the IDF
\item abide by usage policies of the RSP and the IDF
\item use the on-boarding resources to learn about the RSP
\item follow the off-boarding steps if/when appropriate
\item engage in DP0-related discussions via the Community Forum when possible
\item help other DP0 delegates resolve issues when possible
\item include DP0 acknowledgements and references when publishing results
\end{itemize}

Policies regarding the access to and use of the Rubin Observatory data products and services are described in \citeds{RDO-013}.
In the Rubin Data Policy document, DPOL-302 and DPOL-507 describe how only data rights holders may have Rubin Science Platform accounts, and this is why DP0 participants must have data rights.

The simulated data sets used for DP0 are world-public in the language of DPOL-301, which means that the data can be shared with anyone, but does not mean the data will be served publicly.
Anyone may publish results based on analyses of world-public data sets.
(DP0 is therefore a special case: the commissioning data planned for release in DP1 and DP2, and subsequently the survey data, will have a 2-year proprietary period before it is able to be shared with non data rights holders.)

A full list of DP0 acknowledgements and references for delegates who publish results based on their DP0 activities with the DC2 data set in the RSP are provided in the DP0 documentation at \url{dp0-1.lsst.io}. 
For example, DP0 delegates who publish results based on the DC2 simulation are requested to reference the DESC's DC2 paper, \citet{2020arXiv201005926L}.

Policies regarding account usage, security, and off-boarding are described in Section \ref{ssec:res_pol}, and on-boarding resources for DP0 delegates in Section \ref{ssec:res_sup}. 

\subsection{Milestones}
General information on DP0 can be found in \citeds{RTN-001}.
For convenience, those milestones  that are relevant to community participation in DP0 are copied here in \tabref{tab:miles}. 
Jira remains the source of truth for all dates. 
% This file was generated by opsMiles.py do not edit.
\tiny \begin{longtable} {|p{0.3\textwidth}  |r  |r  |r  |r  |p{0.1\textwidth} |} \caption{FY21 Community Engagement Milestones \label{tab:miles}}\\ 
\hline 
{\bf Milestone}&{\bf Jira ID}&{\bf Rubin ID}&{\bf Due Date}&{\bf Level}&{\bf Team}\\ \hline 
Develop a first model for community engagement for DP0.1&\jira{PREOPS-151}&L3-CE-0020&2021-01-31&3&Community Engagement
 \\ \hline
Engage with the community to support shared-risk simulated data distribution to community for science with DP0&\jira{PREOPS-150}&L2-SP-0020&2021-06-30&2&Community Engagement
 \\ \hline
Deliver beta LSST Data Products Documentation (DP0)&\jira{PREOPS-149}&L3-CE-0010&2021-06-30&3&Community Engagement
 \\ \hline
\end{longtable} \normalsize


\section{Delegate Selection Process (DP0.1)}\label{sec:sel}

This section describes how the 300 accounts for DP0.1 were allocated to the science community.

\subsection{Delegate Groups}\label{ssec:sel_grps}

Achieving the goals of DP0 requires a broad diversity in the DP0 delegates, each with their own experience, expertise, or novice perspective, and each representing and reporting back to different parts of the broader community.

The primary goal of this application process is to ensure that the final set of DP0 delegates includes the full range of diversity in the science community, and provides the beneficial opportunities of DP0 participation equitably to them.

To guide the selection process, the groups of DP0 delegates are defined in Table \ref{tab:delegate_groups}, and the \emph{approximate} percentage of delegates in each group estimated.
Representatives in all groups have the same expectations and responsibilities, as in Section \ref{sssec:intro_del_resp}.

\begin{table}[!h]
\centering
\caption{Approximate targeted distribution for diversity-based DP0 delegate groups.}\label{tab:delegate_groups}
\begin{tabular}{lll}
\hline
Group & $\sim$\% & Description \\
\hline \hline
A & 10 & experienced users/builders of the DESC DC2 simulated data set \\
B & 10 & experienced users of science platforms (e.g., RSP, SciServer, DataLab) \\
C & 10 & scientific expertise in Rubin science pillars, deep imaging, big data, etc. \\
D & 10 & representatives of small and/or underserved US institutions and colleges \\
E & 10 & self-identified novice-level learners and students (and students' advisors) \\
F & 10 & representatives of the Chilean astronomy community \\
G & 40 & representatives of the Science Collaborations \\
\hline
\end{tabular}
\end{table}

For Group D, the term 'small and/or underserved US institutions and colleges' refers to, among others: institutions that do not have an R1 or R2 Carnegie classification; institutions that do not grant doctorates in astronomy or physics; institutions that identify as a minority serving institution (MSI), a historically black college or university (HBCU), or a hispanic-serving institute (HSI); or institutions that have other similar aspects.

Applicants will be able to self-identify with multiple groups on the application form (Section \ref{ssec:sel_form}).
Specifically, please note that Chilean astronomers and Science Collaboration members are {\it not} restricted to gaining a delegate position via Groups F and G. 
For example, Stack Club\footnote{\url{https://github.com/LSSTScienceCollaborations/StackClub}} members might self-identify with Group B due to their RSP experience, but could also self-identify with any other group(s).

About one quarter of the accounts in Group G (10\% of the total) will be withheld from the selection process and assigned by the Science Collaborations to representatives engaging in DP0 related activities such as mentoring programs, thereby enabling such programs to exist.

The selection committee described in Section \ref{ssec:sel_comm} will consider all applicants equally for all groups with which applicants self-identify, using the diversity-based selection criteria in Section \ref{ssec:sel_crit}.
The selection process will prioritize diverse representation within the DP0 delegate groups in Table \ref{tab:delegate_groups}, in terms of career stage, global location (to ensure representation of data rights holders from International Programs), language, and underrepresented and minoritized groups.
Section \ref{ssec:sel_form} describes how the application form will collect the relevant diversity information from potential delegates.

\subsection{Selection Committee}\label{ssec:sel_comm}

The delegate selection committee will be comprised of members of the Rubin Observatory Community Engagement Team (CET) and the Rubin Observatory Director's Office.
The selection committee will use the diversity-based selection criteria and process described in Section \ref{ssec:sel_crit}.

The selection committee will continue to seek advisement from the Rubin Observatory Science Advisory Council (SAC) and the LSST Science Collaborations by providing to them a report which summarizes the DP0 delegates identified for DP0.1, including distributions of DP0 delegates across the defined groups (Table \ref{tab:delegate_groups}) and diversity aspects (Sections \ref{ssec:sel_form}).

\subsection{Selection Criteria}\label{ssec:sel_crit}

Diversity in representation is the top priority.
The groups described in Section \ref{ssec:sel_grps} are designed to achieve a diverse population of DP0 delegates in several facets, such as experience, expertise, career level, institution type, geographical location, scientific field, and minority representation.
The selection committee will also consider these aspects of diversity within the groups.

The selection committee will make use of diversity optimization software such as {\tt entrofy} \citep{2019arXiv190503314H} to inform process of identifying DP0 delegates.

Short-form write-in responses will be solicited from applicants so that they may explain their relationship to the groups with which they self-identify.
However, applicants will not be asked to, e.g., provide a robust scientific justification for their participation in DP0, or a detailed plan for their DP0 delegate action (Section \ref{ssec:res_act}).
While it might at first seem reasonable to consider aspects such as "science impact" or "contributed value" in a selection process like this, doing so could run counter to diversity initiatives (i.e., not all applicants will be equally able to express such aspects in language familiar to the committee members).
In a supportive environment with adequate resources (Section \ref{ssec:res_sup}), all DP0 delegates should have an equivalent potential to succeed in their endeavors.


\subsection{Equitable Access}\label{ssec:sel_equity}

There are 4 main ways in which equitable access to DP0 is being supported: broad advertising, a staged application process, on-boarding resources, and flexible time commitments.

Broad advertising is necessary for delegates to self-identify as potential participants, and for reaching potential delegates who are not yet actively participating in Rubin science.
The CET will work together with the Rubin Observatory communications team to ensure the call for applications is broadly advertised. 
The CET will also work with the Rubin directorate to identify individuals in groups A through G, and reach out directly to encourage applications. 

A staged application process is implemented to maximize the inclusion of delegates who would not be able to make a time commitment at the start of DP0.1, and because it is recognized that it might take time to achieve a diverse set of DP0 delegates and to provide equitable access to DP0 (Section \ref{ssec:sel_time}).

On-boarding resources and a supportive community will be generated and maintained by the CET (Section \ref{ssec:res_sup}).
Such resources are recognized as necessary for delegates to acquire the benefits of DP0 participation, regardless of their past experience or expertise.

Flexible time commitments mean that delegates are free to do their DP0-related activities on whatever timescale works for them, and that there is no minimum time commitment or expected ``start date" (Section \ref{sssec:intro_del_resp}).


\subsection{Application Form Contents}\label{ssec:sel_form}

A diversity-based selection process requires that adequate appropriate information be collected from applicants.
The information on the application form will also be used by the CET to help design the delegate resources described in Sections 4.2 and 4.3. 

The application form should be self-explanatory -- applicants should not \emph{have to} refer to this document (or any other) in order to complete the form -- but the form should also link to additional information about DP0.
The application form should include a preamble with the deadline for consideration, the types of data included in DESC DC2, and information for a point-of-contact in case of questions.

Below is a proposed list of the data that should be collected.
All of the drop-down menus should have multiple selections enabled.

Personal Information
\begin{itemize}
\item first name, surname (2 write-in fields)
\item email address (1 write-in field)
\item pronouns (drop-down menu with write-in option)
\item affiliation(s) (drop-down menu with write-in option)
\item city, country (2 write-in fields)
\item first and second languages, preferred correspondence language (3 drop-down menus)
\item career stage (drop-down menu with write-in option)
\item self-identify as a minority (drop-down menu with write-in option)
\end{itemize}

DP0-Related Information
\begin{itemize}
\item science keywords (1 short-form write-in field)
\item Science Collaboration (drop-down menu that includes 'none')
\item delegate group identification (drop-down menu that includes 'none')
\item delegate group relation (one short form write-in field per group, plus one for 'none')
\item group E plus-one request (a yes/no option for applicants to indicate if they are a student/advisor in need of an additional RSP account for their advisor/student).
\end{itemize}

Applicant Confirmations
\begin{itemize}
\item time commitment (yes/no to confirm recognition of minimum expected time commitment)
\item delegate responsibilities (yes/no to confirm recognition of expectations and responsibilities)
\item data rights (yes/no/idk to confirm data rights)
\item origin of data rights for applicants outside the US and Chile (drop-down menu if possible)
\item in-kind proposal team association (drop-down if possible, write-in if not)
\end{itemize}

Regarding the applicant's confirmation of data rights, the form should make clear that all astronomers working in the US and Chile have data rights (including students), and that those individuals should select "yes".

Applicants from outside the US and Chile will indicate the origin of their data rights: either via an agreement between their local institution and the NSF, DOE, AURA, SLAC or LSST Project, or via an in-kind proposal to the US agencies. 
For the former, the selection committee will be able to verify the data rights based on the applicant's name and institution.
For the latter, the selection committee can verify the applicant's data rights status by contacting the relevant in-kind proposals program manager.
Applicants associated with in-kind proposals that are still in the approval process at the time of their DP0 application should be directed to choose "I don't know" and provide the in-kind proposal team association information. 

\subsection{Selection Timeline}\label{ssec:sel_time}

Applications will be primarily reviewed by the selection committee in two stages, which coincide with the projected dates when the DP0.1 and DP0.2 data products will be made available to DP0 delegates in the RSP (Section \ref{ssec:intro_stages}).
These dates are provided in Table \ref{tab:selection_timeline}.
The application due dates are set as late as possible to allow sufficient time for people to learn about DP0 and decide whether to invest time in participating.

Delegates will be notified at least one month in advance of DP0 data becoming available in the RSP (and potentially earlier, if the applications are not oversubscribed).
As described in Section \ref{sssec:intro_del_resp}, delegates need not begin their DP0.1 participation at the time of account activation, and may distribute their time over DP0 at their convenience.

In the case of undersubscription, or in order to more fully include members of the groups identified in Table \ref{tab:delegate_groups} and achieve the diversity-related goals of DP0 participation, the selection committee will also review applications on intermediate timescales (e.g., every couple of months).
The application form will remain open, accumulating submissions, for the duration of DP0.
Applicants will not need to resubmit their application for DP0.2, as all applications received by that time would be considered.

To ensure that at least some accounts are available for new people to join for DP0.2, ~25\% of the 300 total RSP accounts will not be allocated until the second stage (DP0.2).
The DP0.1 data will be available to delegates who join with the second stage.
The following paragraph provides additional motivation for this multi-stage approach to accounts allocation.

The entire duration of DP0 is long, at least 1 year and potentially longer.
It is anticipated that many delegates would not be able or willing to make time commitment for DP0.1 (Section \ref{sssec:intro_del_resp}).
Furthermore, some scientists might be more interested in the DP0.2 data set than DP0.1.
Additionally, achieving diverse and equitable access to DP0 might also take some time: the advertisements must percolate, people must free up time in their schedules or obtain grants to fund their work, and the DP0-related resources and support network will become enriched over time (and thus more attractive to a wider variety of participants).
To avoid the perception that access to any of DP0 is restricted to individuals who can commit to participating at the time of DP0.1, 25\% of the 300 RSP accounts will be reserved for DP0.2.
Any accounts relinquished by DP0.1 participants will be added to this reservation and reallocated for DP0.2.

\begin{table}[!h]
\centering
\caption{Applicants who submit by the dates in column 3 will be notified by the dates in column 4.}\label{tab:selection_timeline}
\begin{tabular}{lllll}
\hline
Stage & \% (of 300) & Applications Due & Delegates Notificatied & Data Available \\
\hline \hline
DP0.1 & 75\%    &  Apr 30 2021   &  May 31 2021  &  Jun 30 2021 \\
DP0.2 & 100\%  &  Jan 31 2022   &  Feb 28 2022  &  Mar 31 2022 \\
\hline
\end{tabular}
\end{table}


\section{Resources for DP0 Delegates}\label{sec:res}

\subsection{Policies for RSP Accounts}\label{ssec:res_pol}

For the duration of DP0, an instance of the RSP will be running at the Interim Data Facility (IDF), which will be deployed on the Google Cloud Platform\footnote{\url{https://www.lsst.org/news/rubin-observatory-will-partner-google-cloud-host-interim-data-facility}}.

Access to the RSP and the DP0 data products and services are limited to data rights holders, and all DP0 delegates must abide by the data rights policies in \citeds{RDO-013}.

All DP0 delegates will be asked to abide by specific policies related to account authentication, security, use, and deactivation.
The full details for these policies will be provided to delegates prior to account creation. 

Delegates may not share accounts or passwords, but will be free (and encouraged!) to provide tours of the RSP or live demonstrations of their work to individuals who are not DP0 delegates.
Account usage policies might include, e.g., best practices for starting and stopping servers, understanding that computational processes that overburden the system might be curtailed, abiding by storage capacity quotas, and following on- and off-boarding steps.

As the total number of accounts will be limited to 300, if applications are oversubscribed then delegates who don't appear to be using their accounts might be periodically contacted and asked to consider voluntarily deactivating to make way for others.
In the case of a high oversubscription for DP0 participation, account holders who do not respond to \emph{multiple} such inquiries \emph{might} have their access suspended to make way for others.
It is understood that delegates might, for example, not have the work-hours to start using their accounts for a few months (due to prior commitments, teaching loads, etc.), and individuals in this situation would need only reply to confirm interest and intent. 

Policies related to account suspension will be clarified for DP0 delegates during on-boarding.
Note that this is a temporary measure for the Data Previews, and not how accounts will be managed during Operations.

\subsection{Support for Delegates}\label{ssec:res_sup}

It is recognized that on-boarding resources and organizational infrastructure provided by Rubin Observatory will be necessary to enable equitable access and benefits for all delegates, including novice science platform users (Section \ref{ssec:sel_equity}). 
However, potential DP0 delegates should be aware that the Rubin Observatory pre-operations team is still building their support resources and infrastructure, and some patience might be required. 

The support resources will be presented for DP0 delegates elsewhere, but would include, e.g., virtual live sessions and recordings of RSP demonstrations, user guides and documentation, and Jupyter notebook tutorials.
The organizational infrastructure will include online open forums for question-and-answer, issue resolution, and general discussion (e.g., Community.lsst.org), as well as regularly scheduled working groups and "office hours" run by Rubin staff.
Opportunities will be provided for DP0 delegates to communicate with the CET about needed resources and support (e.g., periodic surveys, drop-in discussion sessions).

DP0 delegates are responsible for engaging with the resources in order to learn about the RSP and the DP0 data set (Section \ref{sssec:intro_del_resp}).
Creating or enhancing DP0 resources and support infrastructure -- e.g., writing tutorial notebooks, leading a collaborative working group, helping other delegates resolve issues -- are all potential delegate actions (Section \ref{ssec:res_act}).

\subsection{Example Delegate Actions}\label{ssec:res_act}

As described in Section \ref{sssec:intro_del_resp}, the most important responsibility of DP0 delegates is to learn how to do scientific analyses in the RSP, and then take action that informs or enhances the DMS, the RSP, or a Science Collaboration. 

The objective is for these delegate actions to extend the benefits of DP0 beyond the limited number of DP0 participants that Rubin Observatory is able to support, and to maximize the impact of DP0 with respect to the goals of testing and informing development of the RSP and preparing the scientific community for early science (Section \ref{ssec:intro_goals}). 

These actions are envisioned to be simple and enjoyable, and they are completely voluntary and will not be tracked.
Group work is encouraged.
It is left to individual delegates to decide what kind of actions best match their interests and expertise. 

Examples of potential delegate actions include but are not limited to:
\begin{itemize}
\item participate in feedback surveys managed by Rubin Observatory
\item provide a public demo of some science analysis (e.g., Jupyter Notebook)
\item small written reports on activities (e.g., in Community Forum)
\item short presentation on activities during topical DP0-related virtual workshops
\item video demonstrations of "how-to" with the RSP
\item work through an "RSP Test Checklist" provided by Rubin Observatory
\item leading working groups composed of other DP0 delegates
\item helping to resolve reported issues or support other delegate's work
\end{itemize}


\appendix
% Include all the relevant bib files.
% https://lsst-texmf.lsst.io/lsstdoc.html#bibliographies
\section{References} \label{sec:bib}
\renewcommand{\refname}{} % Suppress default Bibliography section
\bibliography{local,lsst,lsst-dm,refs_ads,refs,books}

% Make sure lsst-texmf/bin/generateAcronyms.py is in your path
\section{Acronyms}
\input{acronyms.tex}
% If you want glossary uncomment below -- comment out the two lines above
%\label{sec:acronyms}
%\printglossaries


% \input{checklist}
\end{document}
