\documentclass[DM,authoryear,toc]{lsstdoc}
% lsstdoc documentation: https://lsst-texmf.lsst.io/lsstdoc.html
\input{meta}

% Package imports go here.

% Local commands go here.

%If you want glossaries
% \input{aglossary.tex}
% \makeglossaries

\title{Guidelines for Community Participation in Data Preview 0}

% Optional subtitle
% \setDocSubtitle{A subtitle}

\author{The Community Engagement Team and the Operations Executive Team}

\setDocCurator{Melissa Graham}
\setDocRef{RTN-004}
\setDocUpstreamLocation{\url{https://github.com/rubin-observatory/rtn-004}}

\date{\vcsDate}

% Optional: name of the document's curator
% \setDocCurator{The Curator of this Document}

\setDocAbstract{%

In 2021 the Rubin Observatory began Data Preview 0 (DP0), the first of three data previews during the period leading up to the start of Rubin Operations.
DP0 makes simulated LSST-like data products, which are similar to the future LSST data releases, available in the Rubin Science Platform (RSP).
The goals of DP0 are to serve as an early integration test of the LSST science pipelines and the RSP and to enable the community to prepare to do science with the unprecedented LSST data set. 
To meet these goals, the Rubin Observatory will support a total of 600 RSP accounts for "DP0 delegates": individuals who will represent the science community as early users of the RSP.
DP0 delegates benefit from an accelerated learning experience with the RSP's query, visualization, and analysis tools; are free to pursue their own science with the DP0 data set in the RSP; and have opportunities to provide feedback about their experience to Rubin Observatory.
Of these 600 accounts, 300 were allocated in 2021 as part of Phase 1 (DP0.1) and 300 new accounts will be allocated in 2022 for Phase 2 (DP0.2).
As the number of RSP accounts for community access to DP0 is limited, a set of guidelines for how Rubin Observatory should select the participants is required. 
This document describes the diversity-driven application and selection process for DP0 delegates, and briefly describes some of the resources, support, and activities that will be available to them.
}

% Change history defined here.
% Order: oldest first.
% Fields: VERSION, DATE, DESCRIPTION, OWNER NAME.
% See LPM-51 for version number policy.
\setDocChangeRecord{%
  \addtohist{1}{2020-12-18}{Released for circulation and feedback.}{Melissa Graham}
  \addtohist{2}{2021-01-27}{Released.}{Melissa Graham}
  \addtohist{3}{2022-01-05}{Updated prior to DP0.2.}{Melissa Graham}
}


\begin{document}

% Create the title page.
\maketitle
% Frequently for a technote we do not want a title page  uncomment this to remove the title page and changelog.
% use \mkshorttitle to remove the extra pages

\renewcommand{\thepage}{\arabic{page}}% Arabic numerals for page counter

\setcounter{page}{1}% Start page number

% ADD CONTENT HERE
% You can also use the \input command to include several content files.

\section{Introduction to Data Preview 0}\label{sec:intro}

There are three planned Data Previews (DPs) during the period leading up to the start of Rubin Operations, termed pre-operations,
which will enable the community to access and analyze LSST-like data in the Rubin Science Platform (RSP) on a shared-risk basis.
DP0 is based on simulated data, and started on June 30 2021.
DP1 and DP2 will be based on data taken during Rubin Observatory commissioning, with DP1 currently foreseen to be released by mid-2024 \citedsp{RTN-011}.


\subsection{The Goals of DP0}\label{ssec:intro_goals}

As described in \citeds{RTN-001}, the two goals of DP0 are:
\begin{enumerate}
\item To serve as an early integration test of the Data Management System (DMS) and inform further development of Rubin's LSST Science Pipelines (hereafter LSST Science Pipelines) and the Rubin Science Platform (RSP).
\item To enable the community to learn to use the LSST data products and services, to prepare for early science with commissioning and year 1 data, and to provide feedback to Rubin Observatory. 
\end{enumerate}

In order to achieve these goals, community participation in DP0 has two main components:
(1) performing scientific exercises with the data products and services in the RSP, and
(2) taking action to inform or enhance the performance of the DMS, the RSP, or the science community.

Participants from the science community, "DP0 delegates", are data rights holders\footnote{Data rights holders are US or Chilean scientists, or individuals associated with an International Contribution (in-kind) proposal in progress, as defined in \citeds{RDO-013}.} who represent a diversity of backgrounds.
The term "delegates" reflects the fact that these individuals are representing the science community as learners and as contributors to the achievement of the DP0 goals, as an essential step along the path towards the success of the LSST. 


\subsection{The Total Number of DP0 Delegates}

For provenance, the {\it initial} plan for DP0 (as of Jan 2021) was to have a total of 300 RSP user accounts available at the Interim Data Facility (IDF; the Google Cloud Platform\footnote{\url{https://www.lsst.org/news/rubin-observatory-will-partner-google-cloud-host-interim-data-facility}}) for DP0 delegates, with 75\% allocated for Phase 1 (DP0.1) by Jun 2021, and 25\% reserved for Phase 2 (DP0.2) in 2022 (see Section \ref{ssec:sel_dp01}).

However, the first few months of DP0.1 went so well that Rubin Observatory leadership decided to allocate RSP accounts to all of the DP0.1 applicants, and add 300 new accounts for DP0.2.
Thus, {\bf the total number of RSP accounts supported for DP0 will be 600}, and these 300 new accounts will be allocated by Jun 2022 (see Section \ref{ssec:sel_dp02}).
For reference, 600 RSP accounts for community participants marks an order of magnitude increase in the number of community accounts made available to Stack Club\footnote{\url{https://github.com/LSSTScienceCollaborations/StackClub}} members.

The number of RSP accounts remains limited during the Rubin Observatory construction and pre-operations phase -- and especially for DP0 -- due to the Rubin pre-operations team's limited scope in providing support for data products, software, and services that are still in development.
An order of magnitude increase represents a safe and sustainable scaling up of the Data Management System, which is also still in development, at the IDF.

In addition, there will be $\sim$150 accounts to cover all Rubin Observatory staff members who need one (e.g., Data Production, System Performance, and construction Commissioning, and Community Engagement team members), and to cover members of International Program teams who require DP0 access in order to complete their in-kind deliverables\footnote{Anyone in this situation should contact their Program Manager and Melissa Graham.}.

It is a primary goal of the Rubin pre-operations team to provide RSP access as soon as possible to all data rights holders over the course of the data previews. 
At least by the start of Rubin Observatory Operations, all data rights holders will be able to have RSP accounts.


\subsection{Research Inclusion}\label{ssec:intro_RI}

There are four main ways in which research inclusion for DP0 is being supported: broad advertising, a staged application process, on-boarding resources, and flexible time commitments.

Broad advertising is necessary for delegates to self-identify as potential participants, and for reaching potential delegates who are not yet actively participating in Rubin science.
The Rubin Community Engagement team (CET) works with the Rubin Observatory communications team to ensure the call for applications is broadly advertised. 
The CET also works with the Rubin directorate to identify individuals, especially those at small or under-resourced institutions (Group D, Section \ref{ssec:sel_grps}), and to reach out directly to encourage applications. 

A two-stage application process for DP0.1 and 0.2 is implemented to maximize the inclusion of delegates who would not be able to make a time commitment at the start of DP0.1, and because it is recognized that it might take time to achieve a diverse set of DP0 delegates and to provide equitable access to DP0.
For example, the advertisements must percolate, people must free up time in their schedules or obtain grants to fund their work, and the DP0-related resources and support network will become enriched over time (and thus more attractive to a wider variety of participants).

On-boarding resources and a supportive community are generated and maintained by the CET (see Section \ref{ssec:res_sup} and the Delegates Homepage at \url{dp0-1.lsst.io}).
Such resources are recognized as necessary for delegates to acquire the benefits of DP0 participation, regardless of their past experience or expertise.

Flexible time commitments mean that delegates are free to do their DP0-related activities on whatever timescale works for them, and that there is no minimum time commitment or expected "start date" (Section \ref{ssec:intro_del}).

For DP0.2, a fifth research inclusion strategy was adopted: a DEI postdoc position was added to the CET mid-2022.
Their focus is on understanding the experiences and barriers of DP0 delegates at small or underserved US institutions, serving as a DP0 liaison to those scientists and students, and recommending further initiatives for the CET to support their participation.


\subsection{The DP0 Simulated Data Set}\label{ssec:intro_dc2}

The simulated data set adopted for DP0 was generated by the Dark Energy Science Collaboration (DESC), and is being used under an agreement between the DESC and Rubin.
This data set is the second in a series of data challenges, and is thus named DC2.

As explained in the DESC's DC2 paper \citep{2020arXiv201005926L}, the core science goals that DC2 was developed to address are all extragalactic, such as weak lensing correlations and Type Ia supernova cosmology.
For these reasons, the full DC2 simulation is five years of observations of 300 square degrees of extragalactic regions in six filters with the initial (version 1) baseline survey strategy\footnote{For readers familiar with the term, the operations simulation "OpSim minion\_1016" was used, which represented the initial (version 1) baseline strategy. The 2021 recommendations from the Survey Cadence Optimization Committee (SCOC) about updates to the baseline strategy can be found in \url{ls.st/Document-37922}.}, plus one square degree of a deep drilling field (DDF).
Realistic observing conditions and a baseline survey strategy were applied.

The types of extragalactic objects simulated for DC2 include clusters, galaxies, Type Ia supernovae, and active galactic nuclei (AGN).
The types of galactic objects include stars, about 10\% of which have periodic or non-periodic variability.
DC2 does not include solar system objects, or other types of extragalactic transients such as tidal disruption events or core collapse supernovae.
A full description of the DESC's DC2 simulation can be found in \citep{2020arXiv201005926L}, and in particular the synthetic galaxy catalog is described in \citet{2019ApJS..245...26K}.

A subset of the full DC2 simulation is being used for DP0; this subset does not contain the DDFs and, since AGN were only injected into the DDFs, thus does not contain any simulated AGN.
The image processing outputs that are available as part of DP0 simulate the LSST data release (DR) data products for the wide-fast-deep (WFD) main survey at five years, and are referred to as DR6-WFD (for LSST, DR1 is 0.5 years of data, DR2 is the first year, etc.).
DR6-WFD includes the raw images, processed visit images (PVIs, where a visit is a single 30 second observation), source catalogs for the PVIs, coadded images for each band, and multi-band object catalogs for the coadds.

Further information about DC2 and the DP0 data set can be found in the DP0.1 documentation at \url{dp0-1.lsst.io}.


\subsection{Two Stages: DP0.1 and DP0.2}\label{ssec:intro_stages}

The DP0.1 data set contains the DC2 images and catalogs as processed by the DESC using the LSST Science Pipelines (v19.0.0).
The DP0.1 data products are available in the RSP in their original format as processed by the DESC, which is similar to that of planned LSST data products.
Because the DP0.1 data set release date was June 30 2021, the selection process for the first stage of delegates (Section \ref{ssec:sel_time}) was designed to complete by May 31 2021.
For DP0.1, it is important to note that the processing does not include difference image analysis, and that the catalog photometry is all from the direct images.
Furthermore, the source catalogs for the PVIs are unassociated and forced photometry has not been run on the visit images, which means that there are no time-domain object catalogs or light curves for DP0.1.
The DP0.1 data set will remain available after the release of the DP0.2 data set \citedsp{RTN-001}.

The DP0.2 data set will contain the DC2 images and catalogs reprocessed by Rubin Observatory using a more recent version of the LSST Science Pipelines than DESC used to process the DP0.1 products.
The DP0.2 data products will be available in the RSP in a format that is consistent with the planned LSST data products.
This might include additional data products that are produced by the pipelines as of Sep 2021, for example, difference image analysis (DIA) and DIA photometry are planned as additional DC2 data products for DP0.2.
The projected date by which the DP0.2 data products will be made available to DP0 delegates in the RSP is June 30 2022. 
Therefore, in Section \ref{ssec:sel_time} the second stage of DP0 delegates are scheduled to have their accounts activated by May 31 2022.

Further details about the DP0.1 data set (e.g., table schema, image types) can be found in the DP0.1 documentation at \url{dp0-1.lsst.io}, and similar resources are available for DP0.2 at \url{dp0-2.lsst.io}.

There will also be a few capability differences between the two phases, which might affect some delegates' planned activities.
For example, in DP0.1, the notebook aspect of the RSP offered image access only via the Butler\footnote{The Butler is a middleware component of the DMS for persisting and retrieving image datasets.}, whereas in DP0.2 image access will be available via the Butler, Virtual Observatory (VO) services, and the Portal Aspect of the RSP.
Further details about capability differences can be found in \citeds{RTN-001}.


\subsection{Delegate Benefits and Responsibilities}\label{ssec:intro_del}

DP0 access comes with both benefits and responsibilities.

The anticipated {\bf benefits} for DP0 delegates include, for example:
\begin{itemize}
\item an accelerated learning experience in the Rubin Science Platform
\item expertise that could be useful for grant applications or early science publications
\item the potential to develop, e.g., analysis software, visualization tools, for LSST science 
\item having options to publish or publicize the results of their DP0 work (software, analysis)
\item the ability to advocate for RSP developments beneficial to their scientific field
\end{itemize}

These potential benefits of DP0 participation motivate the focus on research inclusion (Section \ref{ssec:intro_RI}) and the prioritization of diversity in the selection process (Section \ref{sec:sel}).
Documentation resources and support for delegates, provided by Rubin Observatory, are necessary for delegates to obtain these benefits, and they are described in Section \ref{ssec:res_sup}.

The most important {\bf responsibility} is that DP0 delegates understand and abide by the policies for RSP accounts at the IDF (e.g., Acceptable Use, Code of Conduct, Data Rights), which are described in detail in Section \ref{ssec:res_pol}.
Delegates are also expected to learn how to use the RSP well enough to perform some basic scientific exercises, and then take a small action that informs or enhances the DMS, the RSP, or a Science Collaboration. 
Such actions are envisioned to be simple and optional (they will not be tracked), and it will be left to the discretion of individual delegates to decide what kind of actions best match their interests and expertise.
Examples of delegate actions are provided in Section \ref{ssec:res_act}.

The time spent by DP0 delegates will not be tracked and there is no minimum (or maximum), but the estimated minimum time commitment for DP0 delegates to gain at least some of the benefits of participation is approximately 12--36 hours, total.
This would be, e.g., 1--3 hours per week over three months, but this time can be distributed in any manner that is convenient to the delegate, and allocated accounts do not expire.


\section{Delegate Selection Process}\label{sec:sel}

The primary goal of this application process is to ensure that DP0 promotes research inclusion.
More specifically, to ensure that the set of DP0 delegates includes the full range of diversity in the science community, and that the beneficial opportunities of DP0 participation are provided equitably to them.

A key component of Rubin Observatory's Community Engagement Model (\url{ls.st/RTN-006}) is to build a self-sustaining LSST science community that is able to crowd-source solutions. 
Ensuring diversity among the DP0 delegates is key to seeding the distributed expertise that will be the foundation of this community: a network of users from a variety of global locations, institutional types, career stages, and astronomical fields -- a network that includes representatives of groups that are underrepresented in astronomy.


\subsection{Delegate Groups}\label{ssec:sel_grps}

Achieving the goals of DP0 requires a broad diversity in the DP0 delegates, each with their own experience, expertise, or novice perspective, and each representing and reporting back to different parts of the broader community.
To guide the selection process, the groups of DP0 delegates are defined in Table \ref{tab:delegate_groups}, and the \emph{approximate} percentage of delegates in each group estimated.
Representatives in all groups have the same expectations and responsibilities, as in Section \ref{ssec:intro_del}.
These groups are only used during the selection process.

\begin{table}[!h]
\centering
\caption{Approximate targeted distribution for diversity-based DP0 delegate groups.}\label{tab:delegate_groups}
\begin{tabular}{cccl}
\hline
 & \multicolumn{2}{c}{$\sim$\% for DP} & \\
Group & 0.1 & 0.2 & Description \\
\hline \hline
A & 10 & 5 & experienced users/builders of the DESC DC2 simulated data set \\
B & 10 & 5 & experienced users of science platforms (e.g., RSP, SciServer, DataLab) \\
C & 10 & 5 & scientific expertise in Rubin science pillars, deep imaging, big data, etc. \\
D & 10 & 20 & representatives of small and/or underserved US institutions and colleges \\
E & 10 & 30 & novice-level or new-to-Rubin learners and students (and advisors of students) \\
F & 10 & 10 & representatives of the Chilean astronomy community \\
G & 40 & 25 & representatives of the Science Collaborations \\
\hline
\end{tabular}
\end{table}

For Group D, the term 'small and/or underserved US institutions and colleges' refers to, among others: institutions that do not have an R1 or R2 Carnegie classification; institutions that do not grant doctorates in astronomy or physics; institutions that identify as a minority serving institution (MSI), a historically black college or university (HBCU), or a hispanic-serving institute (HSI); or institutions that have other similar aspects.

Applicants will be able to self-identify with {\it multiple groups} on the application form (Section \ref{ssec:sel_form}).
This means that, for example, Chilean astronomers and Science Collaboration members are not restricted to gaining a delegate position via Groups F and G respectively, Stack Club members are not restricted to Group B due to their RSP experience, and so on.

For DP0.1, about one quarter of the accounts in Group G (10\% of the total) were withheld from the selection process and made available for the Science Collaborations to assign to representatives engaging in DP0 related activities such as mentoring programs, thereby enabling such programs to exist.

For DP0.2, because support services have been ramped up and tested by novice and experienced participants during DP0.1, a larger proportion of the slots are being allocated to people who are likely new to Rubin. 


\subsection{Application Form Contents}\label{ssec:sel_form}

A diversity-based selection process requires that adequate and appropriate information be collected from applicants.
The application form is self-contained and self-explanatory -- applicants do not have to refer to this document (or any other) in order to complete the form.
It provides introductory text with the deadline for consideration, the date by which applicants will be notified, and a point of contact; a description of who should and should not use the application form; the goals of DP0 and the types of data included in DESC DC2; who will see the contents of the submitted applications (which are confidential to the CET); and the expectations for DP0 delegates.

Keeping in mind that DP0 is intended to be welcoming to individuals who are new to Rubin and the LSST, the form also provides a description of who has data rights, and links to additional information about DP0, the LSST, and Rubin Observatory.

The DP0 application form questions include:
\begin{itemize}
\item email address (write-in)
\item full name (write-in)
\item institutional affiliation (write-in)
\item city and country of institutional affiliation (write-in)
\item career stage (multiple choice)
\item underrepresented groups (checkboxes)
\item LSST science interests (checkboxes)
\item delegate groups (checkboxes)
\end{itemize}

Applicants who selected "Group E", and are a student (or advisor) who would like an additional account for their advisor (or student), are invited to request a "Plus One" account (see Section \ref{ssec:sel_po}).

Applicants are furthermore requested to attest that they understand that "DP0 Delegates" have responsibilities and that they are willing to take them on, and to confirm that they have Rubin data rights. 
Applicants from outside the US and Chile are requested to indicate which International Team they're a member of, and an "Other" option is provided for individuals who might still be confused about the process.
Rubin staff authenticate all applicants by cross-checking their name and affiliation with, e.g., institutional websites, the International Data Rights Holders List, in-kind project managers.

The application form does {\it not} ask people to, e.g., provide a robust scientific justification for their participation in DP0, or a detailed plan for their DP0 delegate activities.
While it might at first seem reasonable to consider aspects such as "science impact" or "contributed value" in a selection process like this, doing so could run counter to diversity initiatives.
For example, not all applicants will be equally able to express such aspects, especially if they are new to Rubin (regardless of career stage).
In a supportive environment with adequate resources (Section \ref{ssec:res_sup}), all DP0 delegates should have an equivalent potential to succeed in their endeavors, and so a "merit" aspect of this application process is not necessary.


\subsection{Selection Algorithm}\label{ssec:sel_alg}

The groups described in Section \ref{ssec:sel_grps} reserve a portion of the total DP0 slots for applicants with different levels of expertise and experience with Rubin Observatory or similar projects (including novice-level), and for applicants from Chile. 
A diverse population of delegates by career stage, science interest, and from groups that are underrepresented in astronomy on the basis of gender, race, disability, sexual orientation, etc. is also desired for DP0. 

To achieve this, a simple selection algorithm was developed in which every applicant is assigned a weight factor, based on their career stage, science interest, and underrepresented identity.
The weight values are proportional to the fraction of applicants with those characteristics divided by a target fraction for that demographic. 
Target fractions for: (1) career stage are based on promoting junior researchers; (2) science interest are approximately flat across each of the four LSST science pillars and interests outside the pillars; and (3) underrepresented identities are based on the US population.

For each of the Groups A through G (in order; Section \ref{ssec:sel_grps}), a weighted random selection is made from the pool of delegates that indicated membership in the group.
If an applicant was a member of multiple groups, after they are selected they are removed from the pool for the other groups (i.e., they cannot be selected more than once).
This selection process is done 1000 times, with applicants scoring a point every time they're selected.
A score-ranked list of applicants is then used to identify delegates and a waitlist.
The Jupyter Notebook that ran this selection algorithm for DP0.1 is available at \url{ls.st/Document-37722}.

A very similar selection algorithm will be adopted for DP0.2, if the applications are oversubscribed.
In the case of significant undersubscription from any targeted demographic (such as the delegate Groups, career stage, science interest, or underrepresented identity) or if there is a general lack of diversity in the applicant pool, some accounts will remain unassigned and a campaign of targeted advertising or participation invitations will be designed and deployed.
However, if the applications are undersubscribed but the pool is diverse, all applicants would be allocated an account.


\subsection{Plus Ones}\label{ssec:sel_po}

The concept of a "Plus One" for DP0 is that a student (or advisor) can apply to participate in DP0, and then be allocated an additional account for their advisor (or student).
The option for "Plus Ones" was intended to enable teamwork within local institutes and to enable research opportunities for students and for time-constrained advisors.
DP0 delegates allocated a "Plus One" do not need to name them at the time of their account allocation (i.e., in June 2021 or 2022), and can wait until later to have their "Plus One" account allocated (i.e., after the school year starts).
All "Plus Ones" must have Rubin data rights.

For DP0.1, the popularity of this option was underestimated: about $\sim$60 applicants requested a "Plus One" but only 10 accounts within Group E had been reserved for the "Plus Ones". 
By November 2021, all applicants who had requested a "Plus One" account had been invited to provide a name.

For DP0.2, $\sim$50 accounts will be reserved for "Plus Ones".
Part of the reasoning behind increasing the number of "Plus Ones" to allocate is that, anecdotally, for scientists with a limited amount of research time the projects for which they are best able to make progress are the ones being used for student research. 
Furthermore, the scientists with the most limited amount of research time are those with heavy teaching loads, which are more likely to be at small or underserved institutions.


\subsection{Selection Timeline}\label{ssec:sel_time}

The application due dates are set as late as possible to allow sufficient time for people to learn about DP0 and decide whether to invest time in participating.
Delegates will be notified at least one month in advance of DP0 data becoming available in the RSP (and potentially earlier, if the applications are not oversubscribed).
These dates are provided in Table \ref{tab:selection_timeline}.
As described in Section \ref{ssec:intro_del}, delegates need not begin their DP0.1 participation at the time of account activation, and may distribute their time over DP0 at their convenience.

\begin{table}[!h]
\centering
\caption{Applicants who submit by the dates in column 3 will be notified by the dates in column 4.}\label{tab:selection_timeline}
\begin{tabular}{lllll}
\hline
Stage & Applications Due & Delegates Notified & Data Available \\
\hline \hline
DP0.1 & Apr 30 2021   &  May 31 2021  &  Jun 30 2021 \\
DP0.2 & Apr 30 2022   &  May 31 2022  &  Jun 30 2022 \\
\hline
\end{tabular}
\end{table}


\subsection{DP0.1 Allocations}\label{ssec:sel_dp01}

When the DP0.1 selection algorithm was run in May 2021, the total number of DP0 accounts was set at 300 (not 600) and 25\% of them were reserved for DP0.2.
After subtracting the 30 accounts reserved for the Science Collaborations, that left 203 accounts to be allocated for DP0.1.
Ten of those were for "Plus Ones" (see Section \ref{ssec:sel_po}), leaving 193 allocated via the selection process.
For DP0.1, applications were oversubscribed by a factor of $\sim$1.4 with respect to this number of 193, and there was a waitlist.
Over $95\%$ of the initial invitations to participate in DP0 were accepted by applicants, meaning they took action to set up their access to the RSP and attend the kick-off information sessions in July 2021. 
A few additional applicants were invited from the waitlist to compensate for the unresponsive applicants. 

The first few months of DP0.1 went so well that by November 2021, the decision to increase the total number of DP0 accounts to 600 had been made.
At that time, everyone on the waitlist was invited to participate and the option to identify a "Plus One" was extended to all delegates who had requested one.
The rate at which invitees took action to set up their RSP accounts for DP0 was lower than in the early phases.
By December 2021, there were 244 delegates with RSP accounts.


\subsection{New DP0.2 Initiatives}\label{ssec:sel_dp02}

For DP0.2, $\sim$50 RSP accounts will be reserved for special circumstances that further enable science with DP0 and LSST preparatory work within the science community.
These remain to be determined at the discretion of the Rubin Observatory Director for Operations, but examples include accounts for:
\begin{itemize}
\item PIs of successful grant proposals that are LSST related (e.g., NSF AAG)
\item the first cohort of LSST Catalyst fellows (and their "Plus Ones")
\item representatives of the alert broker teams
\item developers of photometric redshift estimators (see \citeds{dmtn-049})
\item "Plus Ones" for Rubin staff advising students on DP0 research projects
\end{itemize}

The last item, "Plus Ones" for Rubin staff, has been added so that staff can avoid any apparent bias or conflict of interest in the selection process that might arise if those students applied, and so that Rubin staff can boost their own DP0-related science productivity with their discretionary science time, which would have a positive impact on DP0 as a whole when that work is shared with other delegates.

Formally, these $\sim$50 accounts would come from the 300 (new) accounts being made available for DP0.2.
However, since as of December 2021 there were only 244 DP0.1 delegates with RSP accounts (out of the 300 DP0.1 accounts; Section \ref{ssec:sel_dp01}), it is likely that these $\sim$50 accounts will absorb the DP0.1 leftovers and that all 300 new DP0.2 accounts will be allocated via the selection algorithm described in Section \ref{ssec:sel_alg}

\subsubsection{DP0.2 Allocations}\label{ssec:sel_dp02_allocated}

For DP0.2, there were 197 applicants with data rights from 23 unique countries, including USA (35\%), Italy (16\%), the UK (8\%), and Chile (8\%).
All applicants were invited to become delegates and participate in DP0.2.

Only 30 applicants requested a "Plus One" account for their student or advisor, a lower fraction than DP0.1 applicants.
All 30 applicants were be invited to extend an invitation to their Plus One.
Career stages are evenly distributed, with students, postdocs, junior faculty/staff and senior faculty/staff comprising about a quarter of the applicants, each.
About 40% of the DP0.2 applicants identify with a group that is underrepresented in astronomy, either on the basis of gender (25\%), race (13\%), and/or other (10\%) bases.
About 52\% of applicants indicated a scientific interest in transients and variables.
In the other science pillars, 40\% indicated interest in dark energy; 23\% in the Milky Way; and 8\% in Solar System.
About 29\% of applicants indicated their primary science interest was beyond the four science pillars.
The Delegate Groups were primarily to be used for account allocation, but they still provide some insight into the demographics of the DP0.2 applicants.
As with the underrepresented identities and science pillars questions, applicants could choose more than one group.

\begin{itemize}
\item Group A: 3\% (experience with DC2)
\item Group B: 14\% (experience with the RSP or other science platforms)
\item Group C: 58\% (experience with survey science)
\item Group D: 8\% (small/underserved US institution)
\item Group E: 43\% (novice, or advisor of a student)
\item Group F: 8\& (Chilean community member)
\item Group G: 50\% (Science Collaboration member)
\end{itemize}


\section{Policies, Resources, and Activities for DP0 Delegates}\label{sec:res}

For a full description of all of the policies, support resources, and activities for DP0 delegates, see the DP0.1 documentation at \url{dp0-1.lsst.io}.


\subsection{Policies for RSP Accounts}\label{ssec:res_pol}

For the duration of DP0, an instance of the RSP will be running at the IDF.
All individuals selected as DP0 delegates will be offered accounts in this deployment of the RSP.

The applicable policies are that DP0 delegates:
\begin{itemize}
\item abide by the Rubin Observatory data policy \citedsp{RDO-013}
\item comply with the Acceptable Use Policy
\item follow the Code of Conduct
\item do not share accounts or passwords
\item follow the on- and off-boarding steps
\item understand the risks inherent in using an in-development platform
\item use the appropriate citations and acknowledgements
\item report bugs and issues when encountered
\end{itemize}

The full details for these policies, and more information about the risks and appropriate use of the RSP, are provided in the DP0 documentation (e.g., \url{dp0-1.lsst.io}) and are reviewed during the initial kick-off info sessions that all new delegates must attend (or watch a recording of).

A full list of DP0 acknowledgements and references for delegates who publish results based on their DP0 activities with the DC2 data set in the RSP are provided in the DP0 documentation (e.g., \url{dp0-1.lsst.io}). 
For example, DP0 delegates who publish results based on the DC2 simulation are requested to reference the DESC's DC2 paper, \citet{2020arXiv201005926L}.


\subsubsection{Rubin Data Rights and DP0}

Access to the RSP and the DP0 data products and services are limited to data rights holders, and all DP0 delegates must abide by the Rubin Data Policy in \citeds{RDO-013}.

In the Rubin Data Policy, items DPOL-302 and DPOL-507 describe how only data rights holders may have Rubin Science Platform accounts, and this is why DP0 participants must have data rights.
The simulated data sets used for DP0 are world-public in the language of DPOL-301, which means that the data can be shared with anyone, but does not mean the data will be served publicly.
Anyone may publish results based on analyses of world-public data sets.
DP0 is therefore a special case: the commissioning data planned for release in DP1 and DP2, and subsequently the survey data, will have a 2-year proprietary period before it is able to be shared with non data rights holders.

Delegate authentication for RSP access (confirming that an applicant has Rubin Data rights) is done by the Rubin staff.
Delegate authorization for RSP access is done by inviting authenticated delegates' to join the {\tt rubin-dp0} GitHub Organization's {\tt Delegates} team. 
Only individuals whose GitHub accounts are associated with that team may log into the RSP at \url{data.lsst.cloud}. 


\subsubsection{Oversubscription and Unused Accounts}

As the total number of accounts will be limited, if applications are oversubscribed then delegates who don't appear to be using their accounts {\it might} be periodically contacted and asked to consider voluntarily deactivating to make way for others.
In the case of a high oversubscription for DP0 participation, account holders who do not respond to \emph{multiple} such inquiries \emph{might} have their access suspended to make way for others.
It is understood that delegates might, for example, not have the work-hours to start using their accounts for a few months (due to prior commitments, teaching loads, etc.), and individuals in this situation would need only reply to confirm interest and intent. 
Note that this is a temporary measure for the Data Previews, and not how accounts will be managed during Operations.


\subsection{Resources and Support for Delegates}\label{ssec:res_sup}

It is recognized that resources and support provided by Rubin Observatory will be necessary to enable equitable access and benefits for all delegates, including novice science platform users (Section \ref{ssec:intro_RI}). 
However, potential DP0 delegates should be aware that the Rubin Observatory pre-operations team is still building their support resources and infrastructure, and some patience might be required. 
A full description of Rubin Observatory's Community Engagement Model, which DP0's resources and support venues are based on, is avialable at \url{ls.st/RTN-006}. 

The support venues and learning resources for DP0 delegates are described in full in the DP0 documentation (e.g., \url{dp0-1.lsst.io}). As a brief summary, they include:
\begin{itemize}
\item instructions and tutorials for using the RSP and the DP0 data set
\item live virtual biweekly seminars ("Delegate Assemblies", all recorded)
\item dedicated Q\&A categories in the Rubin Community Forum
\item a shared GitHub repository for collaboration
\item networking opportunities (working groups, peer mentoring, flash talks)
\item bug reporting and technical assistance via GitHub issues
\end{itemize}

DP0 delegates are responsible for engaging with the resources, and asking questions when needed, in order to learn about the RSP and the DP0 data set and obtain the benefits of DP0 participation (Section \ref{ssec:intro_del}).
Creating or enhancing the DP0 resources and support infrastructure -- e.g., writing tutorial notebooks, leading a collaborative working group, helping other delegates resolve issues -- are all potential delegate actions (Section \ref{ssec:res_act}).


\subsection{Example Delegate Activities}\label{ssec:res_act}

As described in Section \ref{ssec:intro_del}, DP0 delegates are expected to learn how to do scientific analyses in the RSP, and then take action that informs or enhances the DMS, the RSP, or a Science Collaboration. 

The objective is for these delegate actions to extend the benefits of DP0 beyond the limited number of DP0 participants that Rubin Observatory is able to support, and to maximize the impact of DP0 with respect to the goals of testing and informing development of the RSP and preparing the scientific community for early science (Section \ref{ssec:intro_goals}). 

These actions are envisioned to be simple and enjoyable, and they are completely voluntary and will not be tracked.
Group work is encouraged.
It is left to individual delegates to decide what kind of actions best match their interests and expertise. 

Examples of potential delegate actions include but are not limited to:
\begin{itemize}
\item participate in feedback surveys managed by Rubin Observatory
\item work through an "RSP Test Checklist" provided by Rubin Observatory
\item engage in DP0-related discussions via the Community Forum
\item provide a public demo of the RSP or a science analysis (e.g., Jupyter Notebook)
\item post small written reports or recorded demos on DP0 work (e.g., in the Forum)
\item give a short presentation during a Delegate Assembly
\item participate in working groups composed of other DP0 delegates
\item help to resolve reported issues or support other delegate's work
\end{itemize}


\appendix
% Include all the relevant bib files.
% https://lsst-texmf.lsst.io/lsstdoc.html#bibliographies
\section{References} \label{sec:bib}
\renewcommand{\refname}{} % Suppress default Bibliography section
\bibliography{local,lsst,lsst-dm,refs_ads,refs,books}

% Make sure lsst-texmf/bin/generateAcronyms.py is in your path
\section{Acronyms}
\input{acronyms.tex}
% If you want glossary uncomment below -- comment out the two lines above
%\label{sec:acronyms}
%\printglossaries


% \input{checklist}
\end{document}
